\section{Related Work}
\label{sec:related}

\paragraph{2D panoptic segmentation}
2D panoptic segmentation merges semantic and instance segmentation, providing detailed pixel-level parsing of images, capturing both general categories (semantic segmentation)
and individual object identities (instance segmentation) \citep{kirillov2019panoptic}. Since the original task formulation by \citet{kirillov2019panoptic}, a number of works
have been proposed to solve the task \cite{wang2020axial, cheng2020panoptic, mohan2021efficientps, Li_2021_CVPR, Wang_2021_CVPR, Li_2022_CVPR, Kundu_2022_CVPR, Cheng_2022_CVPR, Yu_2022_CVPR, Xu_2023_CVPR, Chen_2023_ICCV, li2023mask, yu2023convolutions},
while more recent approaches \citep{Jain_2023_CVPR} try to unify image segmentation in its entirety.

\paragraph{Single-view 3D reconstruction}
The work by \citet{snavely2006photo} was the first notable attempt at reconstructing 3D scenes from unordered photo collections. Since then, the field of image-based 3D reconstruction
has seen a number of advancements, culminating in the task of single-view 3D reconstruction \citep{choy20163d, wang2018pixel2mesh, mescheder2019occupancy, huang2019perspectivenet, shin20193d, denninger20203d, nie2020total3dunderstanding}.

\paragraph{Shape priors}
\citet{wu2018learning} note that the task of single-view 3D reconstruction is non-deterministic, as there are many 3D shapes that can explain a given single-view input, and propose
to use shape priors to shape the solution space such that the reconstructed shapes are realistic, but not necessarily the ground truth.

\paragraph{3D scene understanding}
The task of 3D scene understanding and panoptic reconstruction is analogous to its 2D counterpart and aims to infer the 3D structure and semantics of a scene, including the 3D layout, object instances, and their 3D shapes from images \citep{dahnert2021panoptic} or noisy geometry \citep{hou20193d, hou2020revealnet}.
\citet{dahnert2021panoptic} propose a method -- henceforth called \emph{Panoptic 3D} -- to jointly solve the tasks of 3D scene understanding and single-view 3D reconstruction by lifting features produced by a 2D backbone into a 3D volume of the camera frustrum, and jointly optimizing for geometric reconstruction as well as semantic and instance segmentation. 

\paragraph{Modality-conditioned shape generation}
3D generative models represent objects in a variety of modalities, including point clouds \citep{achlioptas2018learning, luo2021diffusion}, occupancy grids \citep{mescheder2019occupancy}, meshes \citep{mo2019structurenet}, and signed distance functions \citep{park2019deepsdf}.
Furthermore, these models can also be distinguished by the type of input they take, such as incomplete shapes \citep{dai2017shape}, images \citep{fan2017point}, text \citep{liu2022towards, zhao2023michelangelo}, or other modalities \citep{Zhou_2021_CVPR}. 
Notably, \citet{cheng2023sdfusion} propose \emph{SDFusion}, a 3D object reconstruction method conditioned on images, text and geometrical input.

\paragraph{Datasets}
Notable datasets in the field of panoptic 3D reconstruction include ScanNet \citep{dai2017scannet} and Replica \citep{straub2019replica}, which provide rich annotations for scene understanding tasks.
Another such dataset, 3D-Front \citep{fu20213d}, provides comprehensive coverage of indoor scenes while offering detailed geometric reconstructions as well as semantic and instance segmentation annotations.
The synthetic 3D dataset contains 6,801 mid-size apartments with 18,797 rooms populated by 3D shapes from the 3D-Future \citep{fu20213e} dataset. The dataset's high-quality data acquisition process ensures accurate representations, establishing it as a valuable resource for advancing research in 3D panoptic reconstruction. 

